\pagenumbering{roman} 
\setcounter{page}{1} 
\pagestyle{plain}

%%%%%%%%%%%%%%%%
%%% CREDITOS %%%
%%%%%%%%%%%%%%%%
%\chapter*{Créditos/Copyright}

%Una página con la especificación de créditos/copyright para el proyecto (ya sea aplicación por un lado y documentación por el otro, o unificadamente), así como la del uso de marcas, productos o servicios de terceros (incluidos códigos fuente). Si una persona diferente al autor colaboró en el proyecto, tiene que quedar explicitada su identidad y qué hizo.

%A continuación se ejemplifica el caso más habitual, aunque se puede modificar por cualquier otra alternativa:

%\vspace{1cm}

%\begin{figure}[ht]
%    \centering
%	\includegraphics[scale=1]{images/license.png}
%\end{figure}

%Esta obra está sujeta a una licencia de Reconocimiento -  NoComercial - SinObraDerivada

%\href{https://creativecommons.org/licenses/by-nc-nd/3.0/es/}{3.0 España de CreativeCommons}.

%%%%%%%%%%%%%
%%% FICHA %%%
%%%%%%%%%%%%%
%\chapter*{FICHA DEL TRABAJO FINAL}

%\begin{table}[ht]
%	\centering{}
%	\renewcommand{\arraystretch}{2}
%	\begin{tabular}{r | l}
%		\hline
%		Título del trabajo: & Descriptivo del trabajo\\
%		\hline
%       Nombre del autor: & Nombre y dos apellidos\\
%		\hline
%        Nombre del colaborador/a docente: & Nombre y dos apellidos\\
%		\hline
%        Nombre del PRA: & Nombre y dos apellidos\\
%		\hline
%        Fecha de entrega (mm/aaaa): & MM/AAAA\\
%		\hline
%        Titulación o programa: & Plan de estudios\\
%		\hline
%        Área del Trabajo Final: & El nombre de la asignatura de TF\\
%		\hline
%        Idioma del trabajo: & Catalán, español o inglés\\
%		\hline
%        Palabras clave & Máximo 3 palabras clave\\
%		\hline
%	\end{tabular}
%\end{table}

%%%%%%%%%%%%%%%%%%%
%%% DEDICATORIA %%%
%%%%%%%%%%%%%%%%%%%
%\chapter*{Dedicatoria/Cita}

%Breves palabras de dedicatoria y/o una cita.

%%%%%%%%%%%%%%%%%%%
%%% Agradecimientos %%%
%%%%%%%%%%%%%%%%%%%
%\chapter*{Agradecimientos}



%%%%%%%%%%%%%%%%
%%% RESUMEN  %%%
%%%%%%%%%%%%%%%%
\chapter*{Resumen}
\addcontentsline{toc}{chapter}{Abstract}


Un call-center es el área de una empresa el cuál se encarga de recibir y transmitir llamadas desde o hacia clientes, socios comerciales u otras compañías externas. Debido a la gran cantidad de información que se transfiere en estos centros, resulta una tarea esencial optimizar el tiempo de respuesta para así reaccionar en tiempo real a las peticiones de los clientes y mejorar la percepción que estos tienen sobre la compañía. 

Una manera de mejorar el rendimiento es detectar el tema de las llamadas  mediante técnicas de \textit{machine learning} y dando la posibilidad a la empresa de reaccionar en tiempo real, en función de la temática que se este tratando en cada momento. 



%El documento que se presenta tiene como objetivo final mejorar la operatividad del \textit{call-center}  de una gran empresa, extrayendo mediante técnicas de \textit{machine learning} la temática de las llamadas que se realizan al mismo y dando la posibilidad a la empresa de reaccionar en tiempo real, en función de la temática que se este tratando en cada momento. 

Este nuevo sistema nos permitirá a partir de la transcripción de las llamadas al \textit{call-center} de Telefónica España, descubrir en tiempo real la temática de las mismas. Esta modelización de \textit{topics} se ha realizado utilizando métodos de procesamiento de lenguaje natural y aprendizaje profundo. El sistema realiza la clasificación de las nuevas llamadas en tiempo real, permitiendo a los usuarios visualizar la evolución en la temática de las mismas y generar alertas en base a anomalías.  


TODO Es un borrador volver al  resumen una vez acabado el proyecto.
\onehalfspacing

\vspace{1.5cm}

\textbf{Palabras clave}: ``natural language processing'', ``sentiment analysis'', ``real time'', ``call center'', ``topic modeling'', ``deep learning''